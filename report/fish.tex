\documentclass[11pt]{article}
\usepackage{parskip}
\usepackage{enumitem}
\usepackage[colorlinks]{hyperref}

\title{FISH}
\author{Emil Tullstedt}

\hyphenation{OpenJDK} 

\begin{document}
\maketitle
\tableofcontents

\section{Task Specification}

FISH is a \textit{distributed} file sharing system for local networks using
UDP multicast to search for files shared by a FISH-peer in the network.
\footnote{There is also an implementation where the file searching is done using
a centralized server and persistent TCP connections between that server and
the peers in the network which is not the subject of this report} A file is
shared either individually or by being part
of the shared directory for the FISH client and is assumed to be uniquely
described by filename within the network.

When a file is downloaded from another peer in the network, it will be stored
in a specified download destination directory (by default \texttt{/tmp}). If
the specified download destination directory is equivalent (either by being the
same as or a link to) to the shared directory a file downloaded through FISH
will cause an automatic renewal of files shared from the specific host.

\section{Development Environment}

\subsection{The implementor's hardware \textit{\&} software}

\begin{description}[align=right,labelwidth=2cm]
\item[Computer] Lenovo ThinkPad X250 with Intel Core i3-5010U @ 2.10GHz and 8 GB DDR3
    RAM
\item[Operating System] Fedora 23 running Intel 64 Linux 4.8.13
\item[Java JRE \& JDK] 64-bit OpenJDK 1.8.0 (Java SE 8)
\item[Build system] Gradle 2.5
\end{description}

\subsection{Compiling \textit{\&} Running}
\label{subsec:compile}

\begin{enumerate}
    \item Go to the root directory in the \textit{git repository}
    \item Make sure the \textbf{multicast} repository is the currently checked
        out repository (by running \texttt{git checkout multicast} if necessary)
    \item Run \texttt{\$ gradle build} in order to build the application.
    \item Go to \textit{build/classes/main}
    \item Run \texttt{\$ java is.mjuk.fish.Client
        $<$\textit{file or directory to share}$>$}
\end{enumerate}

The application is now running and should be able to parse the commands
\textbf{find}, \textbf{destination} and \textbf{download} to find a file,
get (without any arguments) or set the download destination (default
\texttt{/tmp}) and download a file to the current download destination.

\section{Description of Technical Implemenation}

\subsection{Overview of Classes}

The entire FISH-implementation is packaged in the \textit{is.mjuk.fish} Java
package with separation of concerns on a class-level. The source files are
available in \textit{$<$gitroot$>$/src/main/java/is/mjuk/fish}.

\begin{description}[align=right,labelwidth=2cm]
    \item[Client] Command-line interface and launcher for the FISH protocol implemenation
    \item[DatagramHandler] Handles multicasting between peers
    \item[Downloader] Thread which download requested files in the background
    \item[Helpers] Static functions and constants that can be used for convenience
    \item[PeerListener] Open a ServerSocket that sends file per request
    \item[UnicastListener] Listens to the peer sharing for information about searches
\end{description}

There is also a single interface \texttt{ConnectorInterface} which is currently
used only by \texttt{DatagramHandler}. The \texttt{ConnectorInterface} defines
a bare minimum for a class running on one or more threads which communicates
over one or more sockets.

\subsection{UDP Multicast for File Discovery}

A peer in the network may ask for files over the UDP Multicast channel
\textit{239.10.10.10}. If a peer (including the original peer) finds the
specified file amongst their shared files, they will respond to the original
query using unicast.

The peer who queried for the file may proceed to ignore the information,
display it to the user or open a TCP-connection to a selected peer in order
to download the file to it's own destination directory. The implemenation
assumes the user never wants to ask for files for the purpose of ignoring the
replies, although that might be abused by a malicious client which is
purposefully trying to incite a DDoS of a network if FISH were widely deployed
within a network or in a more legitimate use-case used by a client which has
successfully downloaded a file and isn't interested by more file listings.


\subsection{Network Packages}

All communication is serialized over the network as UTF-8 byte packages except
for sending the files in themselves (which are transmitted as-is in 4096 byte
chunks)

\subsubsection{Requesting a file listing}

A request for a listing of available sites where a file may be found is sent
over the UDP Multicast channel in the format
\texttt{FIND $<$unicast port$>$ $<$file name (may contain spaces)$>$}.

\subsubsection{Informing about the existance of a file}

A response to a file listing request takes the form
\texttt{$<$TCP Peer port$>$ $<$file name (may contain space)$>$} and is
sent to the UDP unicast port given in a file listing request.

\subsubsection{Requesting a file download}

The requester opens a connection to the requestees TCP peer port followed by
\texttt{DOWNLOAD $<$file name (may contain spaces)$>$}. The requestee responds
by sending a 5 bytes status message as described below

\begin{description}
    \item[\texttt{E\_DNF}] Did not find the file (and won't send)
    \item[\texttt{K\_SEN}] Found the file. Proceeding to send it to requester
\end{description}

If an error occurred (status message beginning with \texttt{E} in case of
future expansion of the FISH protocol) the connection is terminated and the
requester is assumed to proceed to the next possessor of the file (if possible)
\footnote{I.e. the file is available from more peers. The requester should
request the file from the next available peer until successful and this is how
my implementation is handling this}
and not contact the requestee anymore for the same file request. \textit{Note:}
The requester should not assume repeated failure if the user initializes a new
request and should thus ask the requestee again if so happens.

If the file is successfully found and transmission begins (as indicated by
\texttt{K\_SEN}) the file is sent over the same established TCP socket in
incremental chunks of 4096 bytes (statically assigned value) until done.
This operation is assumed successful if the connection is gracefully closed.

\subsection{User Interface}
\label{subsec:ui}

The user communicates with FISH using the FISH Client which is a simple
command-line interface which accepts
\textbf{find}, \textbf{destination} and \textbf{download} as described in
section \ref{subsec:compile}.

\begin{description}[align=right]
    \item[destination] Display current download destination
    \item[destination $<$path$>$] Set a new download destination
    \item[download $<$file$>$] Send a file listing request for the specified
        file and download it automatically
    \item[exit] Close all connections gracefully and die
    \item[find $<$file$>$] Send a file listing request for the specified file
\end{description}

Whenever the client receives the response to a file listing request it displays
the filename, hostname and TCP port to the user (whom may then proceed to
manually download the file using netcat or a similar tool, by simply stripping
the first five bytes of the resulting file).

\section{Running the Application}

See section \ref{subsec:compile} "Compiling \textit{\&} Running" and
section \ref{subsec:ui} "User Interface" for instructions on how to compile and
run the application.

\section{Documentation}

The implementation's documentation is availabe in Javadoc-form on 

\url{https://sakjur.mjuk.is/id2212}

\end{document}
